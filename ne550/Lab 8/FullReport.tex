\documentclass[12pt]{article}

\usepackage{fontspec}
\usepackage{hyperref}
\usepackage[load=addn]{siunitx}
\usepackage{float}
\usepackage[tableposition=top]{caption}

\floatstyle{plaintop}

\setmainfont{Times New Roman}

\title{Semiconductor Detectors}
\author{J.R. Powers-Luhn}
\date{April 27, 2017}

\renewcommand{\baselinestretch}{1.5}
%\setlength{\baselineskip}{0.5em}

\begin{document}
%\linespread{1.5}

\maketitle

\section{Abstract (5\%)}
Here students will tell the reader what was done, why it was done, the important results, and what the significance is. This section is an overview of the entire write up and should be concise and to the point. No figures go into this section but final values do (including uncertainty) and conclusions drawn from these results. This section should be only 250-500 words.

\section{Introduction (10\%)}
This section should cover the basic concepts of what the purpose of the laboratory is. This, in most cases, will require research on the part of the student to figure out what underlying concepts are associated with the assignment. This section is also the theory section. Any references used must be cited and plagiarism will result in automatic F for the assignment. Safe Assign will check all sources, within UTK and online, for text matching any source. The purpose of this section is to allow students to demonstrate an understanding of the basic concepts of the course and the topics covered within the laboratory assignment. This section should be between 3-7 pages.

\section{Experimental (10\%)}
This section needs to give an adequately detailed description of each experiment, including equipment used and short description of method. A bullet point list copied from the laboratory assignment is not acceptable. A general discussion of experimental technique should be included, not a point-by-point discussion of each step in an experiment. Plagiarizing my work will not be tolerated. 

As an example, from Laboratory Assignment 8, experiment setup and Background, we would not list the ~15 steps in sequential order. Indeed, the reader does not need to know that you “Return(ed) the Cs-137 to the GTA.” The appropriate way to report this experiment would be the following (or equivalent): The HPGe detector output was connected to the ORTEC 572A amplifier and its unipolar output was connected to the ORTEC 927 ASPEC MCA. The gain on the amplifier was set to 10, the shaping time to 10 µs, the conversion gain to 14-bits, and a XXXX V bias to the detector. From here, a Cs-137 button source was used to properly set the amplifier gain so that the 662 keV photopeak resided near channel 6,500. Once the settings for the pulse processing chain were properly set, the background radiation environment was measured for ten minutes. One could mention that we ensured that all sources were removed from near-proximity with the HPGe detector, but this is implied from the “background radiation environment.” The goal is to be clear and concise, not verbose. 

\subsection{MCA Linearity}
In order to determine the linearity response of the ORTEC 927 ASPEC MCA, a model BNC BL-2 voltage pulse generator was connected through an ORTEC 572A amplifier to the MCA while the output of the pulse generator was also monitored via an GW INSTEK model GDS-3254 oscilloscope, as indicated in figure \ref{fig:MCAblock}. The output of the MCA was recorded in the Maestro software for several voltage settings, and the channels corresponding to each peak recorded.

\subsection{MCA Dead Time}
In order to determine the dead time of the MCA, the pulse generator was then set to produce a double pulse with a \SI{3}{\micro\second} rise time and a \SI{10}{\micro\second} fall time. Amplifier shaping time was set to \SI{0.5}{\micro\second} with an amplitude of \SI{5.44}{\volt}. The MCA was set to conduct a continuous collection with a 12 bit conversion gain. The spacing between the pulse was adjusted and the threshold at which the MCA was able to distinguish between the pulses (the point at which two peaks appeared on the spectrum) recorded. 

\subsection{HPGe Evaluation}
The MCA was then connected to an HPGe/Preamplifier/Amplifier chain as indicated in figure \ref{fig:hpgeblock}. The amplifier shaping time was set to \SI{10}{\micro\second} with a gain setting of 10 and unipolar output. After a \SI{10}{\minute} background was collected, an assortment of sealed sources were placed at a distance of approximately \SI{10}{\centi\meter} from the detector face. Information regarding the sources is presented in table \ref{tab:hpgesources}. Spectra for these sources were collected in the highest-available number of channels (14 bits) in order to take advantage of the high resolution of the HPGe detector. Sources were counted for \SI{150}{\second} in order to obtain sufficient counts to obtain reasonable statistics.

\begin{center}
\begin{table}
	\centering
	\begin{tabular}{c c c}
		\hline\hline
		Source & Date & Activity \\
		\hline
		Na-22 & April 2015 & \SI{1}{\micro\curie} \\
		Cs-137 & April 2015 & \SI{10}{\micro\curie} \\
		Co-60 & May 2012 & \SI{1}{\micro\curie} \\
		Mn-54 & February 2012 & \SI{10}{\micro\curie} \\
		Cd-109 & May 2015 & \SI{1}{\micro\curie} \\
		Ba-133 & February 2012 & \SI{1}{\micro\curie} \\
		Co-57 & May 2015 & \SI{1}{\micro\curie} \\
		\hline
	\end{tabular}
	\caption{Sources for HPGe Evaluation\label{tab:hpgesources}}
\end{table}
\end{center}

\subsection{HPGe Geometric variation}
A \SI{1}{\micro\curie} Cs-137 source dated February 2012 was used to determine the response of the HPGe detector to sources at various ranges. The source was placed in a holder 

\subsection{CZT Evaluation}

\subsection{Was there another detector?}

This section should be between 1-3 pages.

\section{Results (20\%)}
This section should provide all results from the experiment. This includes graphical representations, when appropriate. This section is where students will draw conclusions from the data and discuss discrepancies and the significance of the results with comparison to accepted values/concepts. Error analysis is important, so make sure all numbers and figures are appropriately represented with uncertainty. When appropriate, compare results with accepted value(s), indicate the discrepancy, and whether the discrepancy is significant or not. Students should identify all sources of error, both systematic and statistical in the experimental technique and methods for improvement. This latter point could also find itself within the conclusion. This section should be as long as needed to present collected/analyzed data and an appropriate level of discussion of results/conclusions through demonstrative arguments.

\section{Conclusion (10\%)}
This section is a final overview of the results and their significance. This should be a summary of section 4, the results section, with an overarching view of the results. This section should be no more than one page.

\section{References (12.5\%)}
References need to be cited in order of appearance in the report. This is NOT a bibliography. I do not want to know what students have read in the past that contributed, but what was specifically used in generation of the report. I would like to see Endnote used with the “numbered” style formatting. EndNote is available for free at \url{https://webapps.utk.edu/oit/softwaredistribution/}. At the end of this document, the Endnote output style is provided for reference.


\end{document}