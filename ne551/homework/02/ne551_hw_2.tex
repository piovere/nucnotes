\documentclass{hw}
\usepackage[version=4]{mhchem}
\usepackage{amsmath}
\usepackage{cancel}

\author{J.R. Powers-Luhn}
\date{2016/09/02}
\title{Homework \#2}

\begin{document}

% \maketitle

\problem{}
	Using both relativistic and non-relativistic kinematics, calculate the kinetic energy of a proton with $\beta$=0.001, 0.01, 0.1, 0.2 and 0.5. Estimate where you start seeing a significant ($ >$5\%) difference between the relativistic and non-relativistic energies.
\solution
	When approaching this relativistically, we know that $ E=\gamma m c^2 $, $ \gamma = \frac{1}{\sqrt{1-\beta^2}} $, and $E=T+mc^2$. 

	\begin{align*}
		E &= T + mc^2\\
		T &= E - mc^2\\
		&= \gamma mc^2 - mc^2\\
		&= (\gamma - 1)mc^2 \\
		&= \left(\frac{1}{\sqrt{1-\beta^2}}-1\right)mc^2\numberthis \label{relativistic}
	\end{align*}

	When considering this from a classical perspective, we know that: 

	\begin{align*}
		T &= \frac{1}{2}mv^2\\
		&= \frac{1}{2}mc^2\frac{v^2}{c^2}\\
		&= \frac{1}{2}\beta^2 mc^2 \numberthis \label{nonrelativistic}
	\end{align*}

	Since the mass of a proton is $ m_{\proton}=938.272MeV $, the nonrelativistic and relativistic kinetic energies are captured in Table \ref{table:betatable}.

	\begin{center}
		\begin{table}[h]
		\centering
			\begin{tabular}{ |c|c|c| }
				\hline
				$ \beta $ & $ T_{classical} $ & $ T_{relativistic} $ \\
				& \eqref{nonrelativistic} & \eqref{relativistic} \\
				\hline
				0.001 & $ 4.69*10^-4 $ & $ 4.69*10^-4 $ \\
				0.01 & $ 4.69*10^-2 $ & $ 4.69*10^-2 $ \\
				0.1 & $ 4.69 $ & $ 4.73 $ \\
				0.2 & $ 18.8 $ & $ 19.3 $ \\
				0.5 & 117 & $ 145 $ \\
				\hline
			\end{tabular}
			\caption {Classical and Relativistic Kinetic Energies of $ \proton $ as a function of $ \beta $}
			\label{table:betatable}
		\end{table}
	\end{center}

	To determine at what energy the error exceeds 5\%, 

	\begin{align*}
		0.05 &> \frac{T_{relativistic} - T_{classical}}{T_{relativistic}} \\
		&> \frac{\left(\frac{1}{\sqrt{1-\beta^2}}-1\right)\cancel{mc^2} - \frac{1}{2}\beta^2 \cancel{mc^2}}{\left(\frac{1}{\sqrt{1-\beta^2}}-1\right)\cancel{mc^2}} \\
		&> 1 - \frac{\beta^2}{2\left(\frac{1}{\sqrt{1-\beta^2}}-1\right)} \\
		\frac{\beta^2}{2\left(\frac{1}{\sqrt{1-\beta^2}}-1\right)} &> 0.95 \\
		\beta^2 &> 1.90*\left(\frac{1}{\sqrt{1-\beta^2}}-1\right) \\
		\intertext{Solving this for $ \beta $ gives:} \\
		\beta &> 0.257
	\end{align*}

\problem{}
	Using relativistic kinematics, calculate the neutron threshold energy for: \ce{n + ^{12}C -> n + 3\alpha, (\alpha=^{4}He)}
\solution
	Specify a 4-momentum vector $ \left(\vec{p}, i\left(E_n + E_{\ce{^{12}C}}\right)\right) $.
	
	Initial state:
	
	\begin{align*}
		\left(\vec{p}, i \left( E_n + E_{\ce{^{12}C}} \right) \right)
		\intertext{Assuming the $ \ce{^{12}C} $ is at rest, $ E_{\ce{^{12}C}} = m\cancel{c^2} $ (for c=1)}
		\left(\vec{p_n}, i \left( E_n + m_{\ce{^{12}C}}\right)\right)^2 &= \left({p_n}^2 - \left({E_n}^2 +{m_{\ce{^{12}C}}}^2 + 2m_{\ce{^{12}C}}E_n\right)\right) \\
		&= \cancel{p_n^2} - \left( \cancel{p_n^2} + m_n^2 + m_{\ce{^{12}C}}^2 + 2 E_n m_{\ce{^{12}C}} \right) \\
	\end{align*}

	Final state (all particles at rest in CM frame):

	\[
	\left( \cancel{\vec{p}}, i\left(m_n + 3m_{\alpha}\right) \right)^2 = -\left( m_n^2 +9m_{\alpha}^2 + 6m_n M_{\alpha} \right)
	\]

	Since 4-momentum is conserved, we can set these as equal to each other:

	\begin{align*}
		\cancel{m_n^2} + m_{\ce{^{12}C}}^2 + 2 E_n m_{\ce{^{12}C}} &= \cancel{m_n^2} + 9m_{\alpha}^2 + 6m_n M_{\alpha} \\
		m_{\ce{^{12}C}}^2 + 2 E_n m_{\ce{^{12}C}} &= 9m_{\alpha}^2 + 6m_n M_{\alpha} \\
		E_n &= \frac{9 m_{\alpha}^2 + 6 m_n m_{\alpha} - m_{\ce{^{12}C}}^2}{2 m_{\ce{^{12}C}}^2} \\
		&= 944.15 MeV
	\end{align*}

\end{document}
