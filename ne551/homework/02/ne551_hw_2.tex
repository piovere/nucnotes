\documentclass{hw}
\usepackage[version=4]{mhchem}

\author{J.R. Powers-Luhn}
\date{2016/08/30}
\title{Homework \#2}

\begin{document}

\problem{1}
	Using both relativistic and non-relativistic kinematics, calculate the kinetic energy of a proton with $\beta$=0.001, 0.01, 0.1, 0.2 and 0.5. Estimate where you start seeing a significant ($ >$5\%) difference between the relativistic and non-relativistic energies.
\solution
	When approaching this relativistically, we know that $ E=\gamma m c^2 $, $ \gamma = \frac{1}{\sqrt{1-\beta^2}} $, and $E=T+mc^2$. 

	\begin{align}
		E &= T + mc^2\\
		T &= E - mc^2\\
		&= \gamma mc^2 - mc^2\\
		&= (\gamma - 1)mc^2
	\end{align}

	When considering this from a classical perspective, we know that: 

	\begin{align}
		T &= \frac{1}{2}mv^2\\
		&= \frac{1}{2}mc^2\frac{v^2}{c^2}\\
		&= \frac{1}{2}\beta^2 mc^2
	\end{align}

\problem{2}
	Using relativistic kinematics, calculate the neutron threshold energy for: \ce{n + ^{12}C -> n + 3\alpha (\alpha=^{4}He)}
\solution
	Got a little chemistry here. Glad I loaded that mchem package.

\end{document}
