\documentclass{hw}
\usepackage[version=4]{mhchem}

\author{J.R. Powers-Luhn}
\date{2016/08/30}
\title{Homework \#2}

\begin{document}

\problem{1}
	Using both relativistic and non-relativistic kinematics, calculate the kinetic energy of a proton with \beta=0.001, 0.01, 0.1, 0.2 and 0.5. Estimate where you start seeing a significant (>5\%) difference between the relativistic and non-relativistic energies.
\solution
	We know that $ E=\gamma m c^2 $ and $ \gamma = \frac{1}{\sqrt{1-\beta^2}} $

\problem{2}
	Using relativistic kinematics, calculate the neutron threshold energy for: \ce{n + ^{12}C -> n + 3\alpha (\alpha=^{4}He)}
\solution
	Got a little chemistry here. Glad I loaded that mchem package.

\end{document}
