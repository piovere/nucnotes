\documentclass{hw}
\usepackage[version=4]{mhchem}
\usepackage{nuc}
\usepackage[load=addn]{siunitx}
\usepackage{amsmath}
\usepackage{cancel}
\usepackage{physics}
\usepackage{booktabs}
\graphicspath{ {images/}}

\DeclareSIUnit\eVperc{\eV\per\clight}
\DeclareSIUnit\year{yr}

\author{J.R. Powers-Luhn}
\date{2016/11/17}
\title{Homework Chapter 10}

\begin{document}

\problem{}
$\ce{^{147}Pm}$ is a pure beta emitter with a \SI{2.6234}{\year} half life. By 
using radioactive seeds, a radioisotope can be evenly distributed in a tumor 
site. Assume a tumor site with a density of \SI{1}{\gram\per\centi\meter^3} 
and a mass of \SI{11}{\gram}. If the plan is to deliver \SI{25}{\gray} of dose 
to the entire prostate, calculate the activity of the $\ce{^{147}Pm}$ at the 
time of implantation into the prostate. Assume it has a biological half life 
similar to $\ce{^{131}I}$.

\solution

\problem{}
\SI{1e-6}{\gram} of $\ce{^{59}Co}$ is placed into the high flux reactor at 
ORNL. After 24 days of irradiation, what is the activity of $\ce{^{60}Co}$ in 
the sample? How many atoms of $\ce{^{59}Co}$ have been lost in that time 
period? Use a flux of \num{1e15} thermal neutrons 
\si{\per\centi\meter^2\per\second}.

\solution

\problem{}
What fluence of neutrons from a DT generator ($\ce{d + t \rightarrow n + 
^{4}He}$) is required to deliver a KERMA of \SI{1}{\gray}?

\solution

\problem{Anderson 10.11}
The linear attenuation coefficient for $\ce{^{60}Co}$ radiation in water is 
\SI{6.5}{\per\meter}.
\begin{enumerate}
    \item Calculate the dose at points at depths \SI{0.01}{\meter}, 
    \SI{0.05}{\meter}, \SI{0.1}{\meter}, \SI{0.2}{\meter} along the central 
    axis for $F$ of \SI{0.8}{\meter}. Assume the maximum dose is 
    \SI{100}{\rad}. Ignore scatter.
    \item Compare your calculations with the measured values in Appendix 11 
    for a $10 \times 10$ \si{\centi\meter} field. Calculate the dose 
    attributable to scatter and the buildup factor at each depth.
\end{enumerate}

\solution

\end{document}