\documentclass{hw}
\usepackage{mhchem}
\usepackage{nuc}
\usepackage[load=addn]{siunitx}
\usepackage{amsmath}
\usepackage{cancel}
\usepackage{physics}
\graphicspath{ {images/}}

\author{J.R. Powers-Luhn}
\date{2016/10/13}
\title{Homework Chapters 5 and 6}

\begin{document}

\problem{Anderson 5.9}
A narrow beam of gamma rays passes through \SI{2.0}{\centi\meter} of lead. The incident beam consists of 30\% \SI{0.4}{\mega\electronvolt} photons and 70\% \SI{1.5}{\mega\electronvolt} photons. What fraction of the incident fluence is transmitted? Use Figure 5.5.

In addition to what's asked for in the question, find the effective attenuation coefficient.

\solution

\problem{Anderson 5.10}
A narrow beam of neutrons passes through \SI{2.0}{\centi\meter} of cadmium. The incident beam consists of 60\% \SI{0.02}{\mega\electronvolt} neutrons and 40\% \SI{0.5}{\mega\electronvolt} neutrons. What fraction of the incident fluence is transmitted? Use the information on Figure 5.6.

\solution

\problem{Anderson 5.14}
Calculate the dose for a \SI{100}{\roentgen} exposure measured in muscle tissue and bone at \SI{18}{\kilo\electronvolt} ($\ce{Mo}-\ce{K_\alpha}$), \SI{140}{\kilo\electronvolt} (\ce{^{99m}Tc}), and \SI{1.25}{\mega\electronvolt} (\ce{^{60}Co}) from the information on Figure 5.14. Assume that electronic equilibrium holds at the point of consideration.

\solution

\problem{}
Calculate the flux of epithermal neutrons needed to deliver a dose rate of \SI{0.1}{\gray\per\second} to muscle (tissue). Use an energy of \SI{0.1}{\mega\electronvolt} to represent the average energy of epithermal neutrons.

\solution

\problem{Anderson 6.4}
What is the angle of scatter and the energy of a Compton electron when the incident photon energy is \SI{140}{\kilo\electronvolt} and the angle of scatter of the photon is \SI{60}{\degree}?

\solution

\problem{}
Calculate the Compton edge energies (max scattered electron energy) for the following isotopes:
\begin{enumerate}
	\item $\ce{^{54}Mn}$
	\item $\ce{^{137}Cs}$
	\item $\ce{^{22}Na}$ (ignore the positron annihilation gammas)
\end{enumerate}

\solution

\problem{}
Using the photon energy from a $\ce{^{137}Cs}$ decay, calculate the following:
\begin{enumerate}
	\item The Klein Nishina total scattering cross section
	\item The total atomic cross section for Compton scattering in lead
	\item The Compton scattering attenuation coefficient in lead
\end{enumerate}

\solution

\problem{}
Calculate the values for $\dv{\sigma_{KN}}{T_e}$ versus $T_e$ assuming an incoming photon energy of \SI{0.5}{\mega\electronvolt}. Calculate the values between $T_e=0$ and $T_e=T_{max}$ in step sizes of \SI{0.02}{\mega\electronvolt}. Plot your results and compare with figure 6.7

\solution

\end{document}