\documentclass{hw}
\usepackage[version=4]{mhchem}
\usepackage{nuc}
\usepackage[load=addn]{siunitx}
\usepackage{amsmath}
\usepackage{cancel}
\usepackage{physics}
\usepackage{booktabs}
\usepackage{hyperref}
\graphicspath{ {images/}}

\DeclareSIUnit\eVperc{\eV\per\clight}
\DeclareSIUnit\year{yr}
\DeclareSIUnit\nucleon{nucleon}

\author{J.R. Powers-Luhn}
\date{2016/11/22}
\title{Homework Chapters 12-14}

\begin{document}

\problem{}
What thickness of \ce{Pb} shielding is needed around a \SI{7.4e13}{\becquerel} point source of \ce{^{60}Co} to reduce the exposure rate to \SI{10}{\milli\rem\per\hour} at a distance of \SI{2}{\meter}?

\solution

\problem{}
The survival of a certain cell line exposed to x-rays is described by $\frac{S}{S_0} = 1 - \left(1 - \exp\left(-0.92 D\right) \right)^2$, with $D$ in \si{\gray}.
\begin{itemize}
    \item What is the RBE for the neutrons (relative to x-rays) for 10\% survival of the cells?
    \item At a lower dose (higher level of survival) is the RBE larger or smaller?
\end{itemize}

\solution

\problem{}
An air monitor located in a mine determines the Radon activity in a \SI{100}{\liter} sample of air. A mine worker is assumed to work \SI{40}{\hour} per week, 50 weeks per year in that environment, and is assumed to breathe at a rate of \SI{0.2}{\meter^3\per\minute}. If all of the Radon in the air is assumed to stay in the body once it is inhaled, and if the ARLI for mine workers is \SI{2e4}{\becquerel\per\year}, what is the maximum allowable Radon activity in the \SI{100}{\liter} sample of air?

\solution

\problem{}
An amount of \ce{^{90}Sr} equal to \SI{1}{\micro\curie} is ingested. What is the total committed dose equivalent? Assume that the \ce{^{90}Sr} settles into the bone marrow and stays in the body (biological half life $\longrightarrow\infty$). Ignore dose from the gamma rays (you'll see that gamma rays are rarely emitted in the \ce{^{90}Sr} decay chain).

\solution

\problem{}
Using the plot of NASA Q values, determine the dose equivalent from \SI{1}{\gray} each of \SI{300}{\mega\electronvolt\per\nucleon} protons and \ce{^{12}C}.

\solution

\end{document}