\documentclass{hw}
\usepackage{mhchem}
\usepackage{nuc}
\usepackage{siunitx}
\usepackage{amsmath}
\usepackage{cancel}
\usepackage{physics}
\usepackage{booktabs}
\graphicspath{ {images/}}

\DeclareSIUnit\eVperc{\eV\per\clight}

\author{J.R. Powers-Luhn}
\date{2016/11/3}
\title{Homework Chapter 8}

\begin{document}

\problem{Anderson 8.3}
Calculate the wavelength of a $\SI{1}{\mega\electronvolt}$ neutron in free space. If a total E of $\SI{1}{\mega\electronvolt}$ is a constant of the motion and the neutron is in a region of a potential well depth of $\SI{10}{\mega\electronvolt}$, what is the wavelength?

\solution
\part
\begin{align*}
	\lambda_{free} &= \frac{h}{\sqrt{2 m_n T}} \\
	&= \frac{
			\SI{4.14E-15}{\electronvolt\second} \times \frac{
															\SI{1}{\mega\electronvolt}
														}
														{
															\SI{1E6}{\electronvolt}
														}
		}
		{
			\sqrt{
				2 \times \SI{939.6}{\mega\electronvolt\per\square\clight} \times \SI{1}{\mega\electronvolt}
			}
		} \\
	&= \SI{2.87E-14}{\meter} \\
	&= \SI{28.7}{\femto\meter}
\end{align*}

\part
\begin{align*}
	\lambda_{well} &= \frac{h}{\sqrt{2 m_n (T + V}} \\
	&= \frac{
			\SI{4.14E-15}{\electronvolt\second} \times \frac{
															\SI{1}{\mega\electronvolt}
														}
														{
															\SI{1E6}{\electronvolt}
														}
		}
		{
			\sqrt{
				2 \times \SI{939.6}{\mega\electronvolt\per\square\clight} \times \SI{11}{\mega\electronvolt}
			}
		} \\
	&= \SI{8.64}{\femto\meter}
\end{align*}

\problem{}
Using the NNDC ENDF evaluations, find (and report) the elastic scattering cross sections for $\SI{0.5}{\mega\electronvolt}$, $\SI{1}{\mega\electronvolt}$, $\SI{5}{\mega\electronvolt}$, and $\SI{10}{\mega\electronvolt}$ neutrons scattering in $\ce{^{207}Pb}$. Use those values to calculate the corresponding mean free paths for elastic scattering.

\solution
Calculations performed in attached code

\begin{tabular}{rrr}
\toprule
T (\si{\mega\electronvolt}) &     $\sigma$ (\si{\per\centi\meter}) &  Mean Free Path (\si{\centi\meter}) \\
\midrule
0.5 &  5.810217 &        5.216310 \\
1.0 &  4.876894 &        6.214590 \\
5.0 &  4.896170 &        6.190124 \\
10.0 &  2.483300 &       12.204686 \\
\bottomrule
\end{tabular}


\problem{Anderson 8.10, change deuterium to $\ce{^{7}Li}$ and change uranium to $\ce{^{206}Pb}$}
Find the number of collisions (nearest integer) necessary to thermalize a beam of \SI{1}{\mega\electronvolt} neutrons with $\ce{^{7}Li}$, $\ce{He}$, $\ce{Be}$, $\ce{C}$, and $\ce{^{206}Pb}$

\solution
Calculations performed in attached code

\begin{tabular}{rlc}
\toprule
A &     Target &  Average Collisions to Thermalize \\
\midrule
7 &       Li-7 &                                67 \\
4 &     Helium &                                41 \\
9 &  Beryllium &                                85 \\
12 &     Carbon &                               111 \\
206 &     Pb-206 &                              1809 \\
\bottomrule
\end{tabular}


\problem{}
A $\SI{14}{\mega\electronvolt}$ neutron scatters elastically off of a $\ce{^{12}C}$ nucleus. The energy given to the $\ce{^{12}C}$ nucleus is \SI{1}{\mega\electronvolt}. What is the scattered angle of the neutron?

\solution
\begin{align*}
	\theta^\prime = \cot^{-1} \frac{1 - A \cos{2 \theta_A^\prime}}{A \sin{ 2 \theta_A^\prime}}
\end{align*}
\begin{align*}
	T^\prime = T \left[ \frac{\cos{\theta^\prime} + \left( A^2 - \sin^2 \theta^\prime \right)^{1/2}}{A + 1} \right]^2
\end{align*}

Substituting in $T=\SI{14}{\mega\electronvolt}$ and $T^\prime=\SI{13}{\mega\electronvolt}$ and using wolframalpha to solve, we get:

\[
	\theta^\prime = \pm \ang{56.22}
\]

\end{document}