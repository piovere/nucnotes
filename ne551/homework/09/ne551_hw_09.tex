\documentclass{hw}
\usepackage{mhchem}
\usepackage{nuc}
\usepackage{siunitx}
\usepackage{amsmath}
\usepackage{cancel}
\usepackage{physics}
\usepackage{booktabs}
\usepackage{hyperref}
\graphicspath{ {images/}}

\DeclareSIUnit\evperc{\eV\per\clight}

\author{J.R. Powers-Luhn}
\date{2016/11/8}
\title{Homework Chapter 9}

\begin{document}

\problem{Anderson 9.3}
Consider the case of a 1 MeV neutron incident on a nucleus of mass number $A = 100$. Using $R=R_0 A^{1/3}=(1.1)10^{-15} A^{1/3}$m, calculate the time required for a single transversal of the nuclear diameter by the neutron. Assume a square well nuclear potential with depth of 10 MeV. Compare this time with an estimate of the compound nucleus lifetime from the uncertainty principle for a resonance with level width $\Delta E=\SI{0.1}{\electronvolt}$

\solution
Please see attached code for work. The two values (\SI{2.24e-22}{\second} for the transit of the nuclear diameter and \SI{2.99e-23}{\second} for the Heisenberg energy and time relationship). These two values vary by 

\problem{}
Look up the thermal neutron cross sections for each stable isotope of Cd, and then using those
values determine the thermal neutron cross section for natural Cd (please include the natural
abundances of each isotope that you use in this calculation)

\solution
The list of stable isotopes was obtained from the ENDF (\url{https://www.nndc.bnl.gov/sigma/index.jsp?as=116&lib=endfb7.1&nsub=10}), and the natural abundance fractions were obtained from NC State (\url{https://www.ncsu.edu/chemistry/msf/pdf/IsotopicMass_NaturalAbundance.pdf}).

Calculations were performed in the attached code. The results were

\begin{tabular}{lrrrrrr}
\toprule
{} &  Abundance &  Cross Section &  Isotope &  Fractional Cross Section Contribution &  Macroscopic Cross Section &  Number Fraction \\
\midrule
0 &       1.25 &          1.000 &      106 &                               0.012500 &               4.914179e+22 &     5.787400e+20 \\
1 &       0.89 &          1.100 &      108 &                               0.009790 &               5.305494e+22 &     4.120628e+20 \\
2 &      12.49 &         11.000 &      110 &                               1.373900 &               5.209030e+23 &     5.782770e+21 \\
3 &      12.80 &         24.000 &      111 &                               3.072000 &               1.126277e+24 &     5.926297e+21 \\
4 &      24.13 &          2.200 &      112 &                               0.530860 &               1.023202e+23 &     1.117200e+22 \\
5 &      12.22 &      20600.000 &      113 &                            2517.320000 &               9.496108e+26 &     5.657762e+21 \\
6 &      28.73 &          0.340 &      114 &                               0.097682 &               1.553570e+22 &     1.330176e+22 \\
7 &       7.49 &          0.075 &      116 &                               0.005618 &               3.367907e+21 &     3.467810e+21 \\
\bottomrule
\end{tabular}


\problem{Anderson 9.9, change material to natural Cd}
What thickness of Cd will remove 95\% of a beam of \SI{100}{\electronvolt} neutrons?

\solution
$$ \phi = \phi_0 \exp{\left(-\Sigma_i * t\right)} $$
$$ t = \frac{-\log \left(\phi / \phi_0 \right) }{\Sigma} $$
$$ \Sigma = \sum_{isotopes} N^{isotope} \sigma_{isotope} $$
$$ N^{isotope} = \frac{A_v * \rho}{\bar{m_{\ce{Cd}}}} f_{isotope} $$
$$ \bar{m_{\ce{Cd}}} = \sum_{isotopes} f_{isotope} m_{isotope} \approx \sum_{isotopes} f_{isotope} A_{isotope} $$

To reduce the beam by 95\% a thickness of \SI{2.57e-2}{\centi\meter} is required

\problem{}
What is the relative probability of production of I-131 with respect to production of Cs-137 in thermal neutron fission of U-235? Use the double-hump curve on Figure 9.18. What is the relative probability of production of Mo-99 with respect to production of Cs-137?

\solution
From Wikipedia (\url{https://en.wikipedia.org/wiki/Fission_product_yield#Ordered_by_yield_.28thermal_neutron_fission_of_U-235.29}), Cs-137 has a yield of 6.0899\% while I-131 has a yield of 2.8336\%. Mo-99 has a yield of 6.1\%. Mo-99 is therefore about as likely to be produced as Cs-137, while I-131 is about half (46.6\%) as likely to be produced as Cs-137.

\end{document}