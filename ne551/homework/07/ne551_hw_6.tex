\documentclass{hw}
\usepackage{mhchem}
\usepackage{nuc}
\usepackage[load=addn]{siunitx}
\usepackage{amsmath}
\usepackage{cancel}
\usepackage{physics}
\graphicspath{ {images/}}

\DeclareSIUnit\eVperc{\eV\per\clight}

\author{J.R. Powers-Luhn}
\date{2016/10/20}
\title{Homework Chapter 7}

\begin{document}

\problem{}
Calculate the threshold for the following photonuclear reactions:
\begin{enumerate}
	\item $\ce{^{12}C}(\gamma,\neutron)\ce{^{11}C}$
	\item $\ce{^{53}Cr}(\gamma,\neutron)\ce{^{52}Cr}$
	\item $\ce{^{105}Pd}(\gamma,\neutron)\ce{^{104}Pd}$
	\item $\ce{^{183}W}(\gamma,\neutron)\ce{^{182}W}$
	\item How do these thresholds compare with what you would expect for a typical binding energy of a nucleon in a nucleus?
\end{enumerate}

\solution

\begin{align*}
	SE_b &= \Delta E_b + \Delta E_Y - \Delta E_X \\
	\left( h \nu \right)_{min} &= BE_b \left( 1 + \frac{BE_b}{2 M_X c^2} \right) \\
	&= \left(SE_b + \cancel{E^*_Y} \right) \times \left( 1 + \frac{\left(SE_b + \cancel{E^*_Y} \right)}{2 M_X c^2} \right)
\end{align*}

\part \SI{34.395}{\mega\electronvolt}
\part \SI{8.5801}{\mega\electronvolt}
\part \SI{7.3485}{\mega\electronvolt}
\part \SI{6.2997}{\mega\electronvolt}

\part They are slightly higher than expected due to the requirement that the daughter nucleus have some positive momentum (and therefore energy)

\problem{Anderson 7.4}
Suppose a \SI{140}{\kilo\electronvolt} photon undergoes photoelectric effect in a lead sheet with a $K$-shell electron.
\begin{enumerate}
	\item What is the kinetic energy liberated?
	\item If it is assumed that this is all photoelectron kinetic energy, calculate the electron momentum and the photon momentum and compare the two.
\end{enumerate}

\solution

\part From Appendix 4, the K-shell binding energy is \SI{88.004}{\kilo\electronvolt}.

\begin{align*}
T_{lib} &= E_{\gamma} - BE \\
&=\SI{140}{\kilo\electronvolt} - \SI{88.004}{\kilo\electronvolt} \\
&=\SI{51.996}{\kilo\electronvolt}
\end{align*}

\part


\begin{align}
p_{\gamma} &= \frac{h \nu}{c} \\
&= \SI{140}{\kilo\electronvolt\per\clight} \\
p_{\electron} &= \sqrt{2 m E} \\
&= \sqrt{2 \times \SI{511}{\kilo\electronvolt\per\clight^2} \times \SI{51.996}{\kilo
\electronvolt}} \\
&= \SI{230.521}{\kilo\electronvolt\per\clight} \\
\end{align}

\problem{Anderson 7.11}
Given that the mass attenuation coefficient for $\ce{^{63}Cu}$ is \SI{0.474}{\meter^2\per\kilo\gram} at \SI{40}{\kilo\electronvolt} (photoelectron dominates) and \SI{0.0042}{\meter^2\per\kilo\gram} at \SI{2}{\mega\electronvolt} (incoherent scatter dominates), estimate the coefficient for $\ce{^{56}Fe}$ at these energies.

\solution
Since these are neighboring elements, the general form $$ \left( \frac{\mu}{\rho} \right)_{\ce{Fe}} \approx \frac{\left( Z^n / M_m \right)_{\ce{Fe}}}{\left(Z^n/M_m\right)_{\ce{Cu}}} \left(\frac{\mu}{\rho} \right)_{\ce{Cu}} $$ applies. For photoelectron we use $n=4$, while for incoherent scattering we use $n=1$.

For simplicity, use $M_m \approx A_m$

\begin{align*}
M_{\ce{^{56}Fe}} &\approx \SI{56}{\amu} \\
Z_{\ce{^{56}Fe}} &= 26 \\
M_{\ce{^{63}Cu}} &\approx \SI{63}{\amu} \\
Z_{\ce{^{63}Cu}} &= 29
\end{align*}

\part $E=\SI{40}{\kilo\electronvolt}$
\begin{align*}
\left( \frac{\mu}{\rho} \right)_{\ce{Fe}} &\approx \frac{26^4/56}{29^4/63}*\SI{0.474}{\meter^2\per\kilogram}\\
&\approx \SI{0.345}{\meter^2\per\kilogram}
\end{align*}

\part $E=\SI{2}{\mega\electronvolt}$
\begin{align*}
\left( \frac{\mu}{\rho} \right)_{\ce{Fe}} &\approx \frac{26/56}{29/63}*\SI{0.0042}{\meter^2\per\kilogram}\\
&\approx \SI{0.00424}{\meter^2\per\kilogram}
\end{align*}


\problem{}
Go to the NIST XCOM webpage and find the photon energies where the photoelectric effect and Compton scattering (incoherent scattering) have the same magnitudes for:
\begin{enumerate}
	\item Carbon
	\item Aluminum
	\item Copper
	\item Tungsten
	\item Uranium
\end{enumerate}

\solution

\part \SI{22}{\kilo\electronvolt}
\part \SI{52}{\kilo\electronvolt}
\part \SI{130}{\kilo\electronvolt}
\part \SI{463}{\kilo\electronvolt}
\part \SI{677}{\kilo\electronvolt}

\end{document}