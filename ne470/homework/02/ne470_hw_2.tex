\documentclass{hw}
\usepackage{nuc}
\usepackage[version=3]{mhchem}

\author{J.R. Powers-Luhn}
\date{2016/09/08}
\title{Homework \#2}

\begin{document}

\problem{2-7}
	Suppose we consider a beam of neutrons incident upon a thin target with an intensity of $ 10^{12} \frac{neutrons}{cm^2 s} $. Suppose further that the total cross section for the nuclei in this target is $ 4 b $. Using this information, determine how long one would have to wait, on the average, for a given nucleus in the target to suffer a neutron interaction.

\solution
	We know that our reaction rate is equal to $ intensity * cross\ section $. Our mean time before a reaction is the inverse of the rate:
	
	\[
		\frac{1}{intensity*cross\ section} = 2.5*10^{11}s
	\]

\problem{2-11}
	Using the data from BNL-325, compute the mean free paths of neutrons with the following energies in the specified materials:
	\begin{enumerate}
		\item $ 14 MeV $ neutrons in air, water, and uranium (characteristic of thermonuclear fusion neutrons),
		\item $ 1 MeV $ neutrons in air, water, and uranium (fast breeder reactor neutrons), and
		\item $ 0.05 eV $ neutrons in air, water, and uranium (thermal reactor neutrons).
	\end{enumerate}

\solution
	\begin{table}[h]
		\centering
		\begin{tabular}{ |c|c|c|c| }
			\hline
			& $ 14MeV $ & $ 1MeV $ & $ 0.05eV$ \\
			\hline
			U & $ 3.54cm $ & $ 2.91cm $ & $ 1.42cm $ \\
			Water & $ 10.04cm $ & $ 1.79cm $ & $ 0.53cm $ \\
			Air & $ 1.27*10^2 cm $ & $ 5640 cm $ & $ 2041 cm $ \\
			\hline
		\end{tabular}
	\end{table}

\problem{2-12}
	Determine the kinetic energy at which the wavelength of a neutron is comparable to:
	\begin{enumerate}
		\item the diameter of a nucleus,
		\item an atomic diameter,
		\item the interatomic spacing in graphite, and
		\item the diameter of a nuclear reactor core.
	\end{enumerate}
	(Only rough estimates are required.)

\solution
	The deBroglie wavelength of a particle is expressed by:
	\[ \lambda = \frac{h}{\sqrt{2Tm}} \]
	from which we derive:
	\[ T = \frac{h^2}{2m{\lambda}^2} \]
	
	\begin{table}[h]
		\begin{tabular}{ |c|c|c| }
			\hline
			Object & $ \lambda $ & T \\
			\hline
			the diameter of a nucleus & 15 fm & 3.63 MeV \\
			an atomic diameter & 350 pm & $ 6.68 * 10^{-9} MeV $ \\
			the interatomic spacing in graphite & 0.142 nm & $ 4.06*10^{-9} MeV $ \\
			the diameter of a nuclear reactor core & 1 m & $ 8.18*10^{-28} MeV $ \\
			\hline
		\end{tabular}
	\end{table}

\problem{2-15}
	Using the Maxwell-Boltzman distribution $ M(V,T) $, calculate the most probable energy of the nuclei characterized by such a distribution. Also calculate the average thermal energy of these nuclei.

\solution


\problem{2-20}
	Determine the fission-rate density necessary to produce a thermal power density of $ 400 kW/liter $ (typical of a fast breeder reactor core). Assume that the principle fissile isotope is \Pu{239}.

\solution
	Each fission of \Pu{239} produces $ 211.5 MeV $. Therefore:
	
	\begin{align*}
		TPD &= \frac{E}{fission} * FRD \\
		FRD &= \frac{TPD}{\frac{E}{fission}} \\
		&= \frac{400 \frac{J}{s*L}}{211.5 \frac{MeV}{fission} / 6.2415*10^{12} \frac{MeV}{J}} \\
		&= 1.18*10^{16} \frac{fissions}{L*s}
	\end{align*}


\end{document}