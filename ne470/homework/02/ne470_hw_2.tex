\documentclass{hw}
\usepackage{nuc}
\usepackage[version=3]{mhchem}

\author{J.R. Powers-Luhn}
\date{2016/09/08}
\title{Homework \#2}

\begin{document}

\problem{2-7}
	Suppose we consider a beam of neutrons incident upon a thin target with an intensity of $ 10^{12} \frac{neutrons}{cm^2 s} $. Suppose further that the total cross section for the nuclei in this target is $ 4 b $. Using this information, determine how long one would have to wat, on the average, for a given nucleus in the target to suffer a neutron interaction.

\solution


\prolem{2-11}
	Using the data from BNL-325, compute the mean free paths of neutrons with the following energies in the specified materials:
	\begin{enumerate}
		\item $ 14 MeV $ neutrons in air, water, and uranium (characteristic of thermonuclear fusion neutrons),
		\item $ 1 MeV $ neutrons in air, water, and uranium (fast breeder reactor neutrons),
		\item $ 0.05 eV $ neutrons in air, water, and uranium (thermal reactor neutrons).
	\end{enumerate}

\solution


\problem{2-12}
	Determine the kineteic energy at which the wavelength of a neutron is comparable to:
	\begin{enumerate}
		\item the diameter of a nucleus,
		\item an atomic diameter,
		\item the interatomic spacing in graphite, and
		\item the diameter of a nuclear reactor core.
	\end{enumerate}
	(Only rough estimates are required.)

\solution


\problem{2-15}
	Using the Maxwell-Boltzman distribution $ M(V,T) $, calculate the most probable energy of the nuclei characterized by such a distribution. Also calculate the average thermal energy of these nuclei.

\solution


\problem{2-20}
	Determine the fission-rate density necessary to produce a thermal power density of $ 400 kW/liter $ (typical of a fast breeder reactor core). Assume that the principle fissile isotope is \Pu{239}.

\solution



\end{document}