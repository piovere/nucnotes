\documentclass{hw}
\usepackage[version=3]{mhchem}

\author{J.R. Powers-Luhn}
\date{2016-09-01}
\title{Homework #1}

\begin{document}

\problem{2-1}
	What target isotope must be used for forming the compound nucleus $ \ce{^{24}_{11}Na} $ when the incident projectile is:
	\begin{enumerate}
		\item a neutron
		\item a proton
		\item an alpha particle?
	\end{enumerate}
\solution
	\part A neutron will increase the mass number, $ A $, by one, but leave the element number, $ Z $, unchanged. Therefore, the answer is a lighter isotope of Neon: \ce{^{23}_{11}Na}
	\part Capturing a proton increases both the mass number and element number by one: \ce{^{23}_{10}Ne}
	\part Capturing an $ \alpha $ particle increases the mass number by four and the element number by two: \ce{^{20}_{8}O}

\problem{2-4}
	A fission product of very considerable importance in thermal reactor operation is $ 

\end{document}