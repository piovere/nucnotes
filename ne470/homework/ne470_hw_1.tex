\documentclass{hw}
\usepackage[version=3]{mhchem}
\usepackage{mychemistry}

\author{J.R. Powers-Luhn}
\date{2016-09-01}
\title{Homework #1}

\begin{document}

\problem{2-1}
	What target isotope must be used for forming the compound nucleus $ \ce{^{24}_{11}Na} $ when the incident projectile is:
	\begin{enumerate}
		\item a neutron
		\item a proton
		\item an alpha particle?
	\end{enumerate}
\solution
	\part A neutron will increase the mass number, $ A $, by one, but leave the element number, $ Z $, unchanged. Therefore, the answer is a lighter isotope of Neon: \ce{^{23}_{11}Na}
	\part Capturing a proton increases both the mass number and element number by one: \ce{^{23}_{10}Ne}
	\part Capturing an $ \alpha $ particle increases the mass number by four and the element number by two: \ce{^{20}_{8}O}

\problem{2-4}
	A fission product of very considerable importance in thermal reactor operation is $ \ce{^{135}Xe}$, which has an enormous thermal absorption cross section of $ 2*10^6 b $. This nuclide can be produced either directly as a fission product or by beta decay of $ \ce{^{135}I} $, as indicated by the radioactive chains below:
	\begin{rxn}
		\reactant[,start]fission \arrow \ce{^{135}I} \arrow \ce{^{135}Xe}
	\end{rxn}
	Write the rate equations describing the concentration of $ \ce{^{135}I} $ and $ \ce{^{135}Xe} $ in a nuclear reactor. Then assuming a constant production rate of these isotopes from fission and transmutation rate by neutron capture, determine the steady-state or saturated concentration of $ \ce{^{135}Xe} $.
\solution
	Holy hell that was really hard! Like, just typing it!

\problem{2-6}
	Boron is a common material used to shield against thermal neutrons. Estimate the thickness of boron required to attenuate an incident thermal neutron beam to 0.1\% of its intensity. (Use the thermal cross section data in Appendix A.)
\solution
	This one was quite a bit easier.

\problem{2-8}
	A free neutron is unstable against beta decay with a half-life of 11.7m. Determine the relative probability that a neutron will undergo beta-decay before being absorbed in an infinite medium. Estimate this probability for a thermal neutron in $ \ce{H_{2}O} $.
\solution
	Not too bad. Did have to break out the \\ce, though

\problem{2-10}
	How many mean free paths thick must a shield be designed in order to attenuate an incident neutron beam by a factor of 1000?
\solution
	From \textit{2-27}

\end{document}
