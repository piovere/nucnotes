\documentclass{hw}
\usepackage{amsmath}
\usepackage[load=addn]{siunitx}
\usepackage{nuc}
\usepackage[super]{nth}

\author{J.R. Powers-Luhn}
\date{2018/03/02}
\title{Homework \#2}

\begin{document}

% \maketitle

\problem{}
	We would like to take an asorption spectrum of a sample that we believe has its absorption maximum at \SI{200}{\nano\meter}. Which of the following materials should our cuvette be made of: borosilicate glass, polystyrene, or quartz? Explain why your choice is the best option.
\solution
    

\problem{}
    Why are mirrors preferred over lenses for imaging in many spectroscopic instruments that must cover multiple wavelengths?
\solution


\problem{}
    What performance characteristics of a monochromator are affected when only the grating groove density is changed?
\solution

\problem{}
    A ray in air (n=\num{1.33}) is incident on a block of sapphire (n=\num{1.77}) at a \ang{40} angle from the normal to the glass surface. At what angle relative to the normal will the ray be transmitted through the glass?
\solution

\problem{}
    What is a birefringent crystal? How does a birefringent crystal work if you send light through it? What is an example of a birefringent crystal?
\solution

\problem{}
    List the four types of lenses and identify them as either converging (C) or diverging (D).
\solution

\problem{}
    A thin biconvex lens of refractive index \num{1.47} and diameter of \SI{50.8}{\milli\meter} has radii of curvature of R1=\SI{1}{\centi\meter} and R2=\SI{0.5}{\centi\meter}.
    \begin{enumerate}
        \item Find the focal point of the lens
        \item If the object is placed \SI{2}{\centi\meter} from the lens, where is the image?
        \item What is the f/\# of the lens?
        \item Calculate the NA of the lens
    \end{enumerate}
\solution

\problem{}
    What is the definition of an optical aberration? Also name the two types of optical aberrations.
\solution

\problem{}
    A grating has a groove density of \num{3600} grooves per \si{\milli\meter}. If the incident beam strikes the grating at an angle of \ang{30},
    \begin{enumerate}
        \item What diffraction angle will the first order of \SI{240}{\nano\meter} appear?
        \item What diffraction angle will the first order of \SI{350}{\nano\meter} appear?
        \item What can we conclude about the relationship between diffraction angle and incident wavelength from the answers you calculated in a and b?
        \item What wavelength in the \nth{2} order overlaps with the \SI{350}{\nano\meter} \nth{1} order beam?
        \item What is the free spectral range for the \nth{1} order at \SI{600}{\nano\meter}?
    \end{enumerate}
\solution

\problem{}
    For a fiber optic probe with core and cladding refractive indices of \num{1.50} and \num{1.48}, respectively, and $\theta_i = \ang{28}$, calculate:
    \begin{enumerate}
        \item $\theta_r$
        \item NA
    \end{enumerate}
\solution

\problem{}
    A monochromator has the following specifications:
    \begin{itemize}
        \item reciprocal linear dispersion = \SI{1.5}{\nano\meter\per\milli\meter}
        \item focal length = \SI{320}{\milli\meter}
        \item f/\# = \num{4.6}
        \item grating size: \num{68} $\times$ \SI{68}{\milli\meter}
        \item groove density = \SI{1800}{grooves \per\milli\meter}
    \end{itemize}
    Calculate the following at \SI{633}{\nano\meter} assuming the \nth{1} order is used:
    \begin{enumerate}
        \item Angular dispersion
        \item Linear dispersion
        \item Slit width to obtain a \SI{5}{\nano\meter} geometric spectral bandpass
    \end{enumerate}
\solution

\end{document}
