\documentclass{hw}
\usepackage{amsmath}
\usepackage[load=addn]{siunitx}
\usepackage[super]{nth}

\sisetup{
    exponent-to-prefix = true,
    zero-decimal-to-integer
}

\title{Homework \#2}
\author{J.R. Powers-Luhn}
\date{2018/03/02}

\begin{document}

% \maketitle

\problem{}
    We would like to take an asorption spectrum of a sample that we believe has its absorption maximum at \SI{200}{\nano\meter}. Which of the following materials should our cuvette be made of: borosilicate glass, polystyrene, or quartz? Explain why your choice is the best option.
\solution
    In order to measure the absorption in our sample, we need to ensure that our cuvette does not absorb our signal. Borosilicate has a very low transmission fraction below \SI{300}{\nano\meter} or so, so it is a poor choice. Similarly, polystyrene has a strong, broad absorption peak at \SI{250}{\nano\meter}, making it unsuitable for this cuvette.

\problem{}
    Why are mirrors preferred over lenses for imaging in many spectroscopic instruments that must cover multiple wavelengths?
\solution
    Lenses may absorb wavelengths of interest, depending on their materials. Mirrors cover a broad wavelength range, e.g. \SIrange[scientific-notation = engineering]{250}{20000}{\nano\meter} in the case of Alumninum-backed mirrors.

\problem{}
    What performance characteristics of a monochromator are affected when only the grating groove density is changed?
\solution
    This affects the resolution of the monochromator. More groove density means that the spacing between the grooves is smaller. This means that the resolution, $R = \frac{\lambda}{\Delta \lambda} = n N$, increases.

\problem{}
    A ray in air (n=\num{1.33}) is incident on a block of sapphire (n=\num{1.77}) at a \ang{40} angle from the normal to the glass surface. At what angle relative to the normal will the ray be transmitted through the glass?
\solution
    \begin{align*}
        n_1 \sin \theta_1 &= n_2 \sin \theta_2 \\
        \theta_2 &= \arcsin \frac{n_1 \sin \theta_1}{n_2} \\
        &= \arcsin \frac{\num{1.33} \times \sin \ang{40}}{\num{1.77}} \\
        \theta_2 &= \ang{28.9}
    \end{align*}

\problem{}
    What is a birefringent crystal? How does a birefringent crystal work if you send light through it? What is an example of a birefringent crystal?
\solution
    A birefringent crystal is a crystal made of a material whose index of refraction depends on the polarization of the incident light. When light is sent through the crystal, it splits into two different beam paths and \textbf{double refracts}. An observer on the other side will see two images, offset by a fixed amount. An example of a birefringent crystal is calcite.

\problem{}
    List the four types of lenses and identify them as either converging (C) or diverging (D).
\solution
    There are six types of lenses:
    \begin{itemize}
        \item Biconvex: converging
        \item Plano-convex: converging
        \item Positive meniscus: converguing
        \item Negative meniscus: diverging
        \item Plano-concave: diverging
        \item Biconcave: diverging
    \end{itemize}

\problem{}
    A thin biconvex lens of refractive index \num{1.47} and diameter of \SI{50.8}{\milli\meter} has radii of curvature of $R_1=\SI{1}{\centi\meter}$ and $R_2=\SI{0.5}{\centi\meter}$.
    \begin{enumerate}
        \item Find the focal point of the lens
        \item If the object is placed \SI{2}{\centi\meter} from the lens, where is the image?
        \item What is the f/\# of the lens?
        \item Calculate the NA of the lens
    \end{enumerate}
\solution
    \part 
        \begin{align*}
            \frac{1}{f} &= \left( n - 1 \right) \left(\frac{1}{R_1} - \frac{1}{R_2} \right) \\
            &= \left(1.47 - 1 \right)\left( \frac{1}{\SI{1}{\centi\meter}} - \frac{-1}{\SI{0.5}{\centi\meter}} \right) \\
            \frac{1}{f} &= \SI{1.41}{\per\centi\meter} \\
            f &= \SI{7.09}{\milli\meter}
        \end{align*}
    \part
        \begin{align*}
            \frac{1}{f} &= \frac{1}{s_1} + \frac{1}{s_2} \\
            \frac{1}{s_2} &= \frac{1}{f} - \frac{1}{s_1} \\
            s_2 &= \left( \frac{1}{\SI{7.09}{\milli\meter}} - \frac{1}{\SI{20}{\milli\meter}} \right)^{-1} \\
            s_2 &= \SI{1.10}{\centi\meter}
        \end{align*}
        The image will be \SI{1.10}{\centi\meter} behind the lens.
    \part
        \begin{align*}
            f/\# &= \frac{f}{D} \\
            &= \SI{7.09}{\milli\meter} / \SI{50.8}{\milli\meter} \\
            &= \num{0.140}
        \end{align*}
    \part
        \begin{align*}
            f/\# &\approx \frac{1}{2 N.A.} \\
            N.A. &\approx \frac{1}{2 f/\#} \\
            &\approx \left( 2 * \num{0.140} \right) \\
            &\approx \num{3.58}
        \end{align*}

\problem{}
    What is the definition of an optical aberration? Also name the two types of optical aberrations.
\solution

\problem{}
    A grating has a groove density of \num{3600} grooves per \si{\milli\meter}. If the incident beam strikes the grating at an angle of \ang{30},
    \begin{enumerate}
        \item What diffraction angle will the first order of \SI{240}{\nano\meter} appear?
        \item What diffraction angle will the first order of \SI{350}{\nano\meter} appear?
        \item What can we conclude about the relationship between diffraction angle and incident wavelength from the answers you calculated in a and b?
        \item What wavelength in the \nth{2} order overlaps with the \SI{350}{\nano\meter} \nth{1} order beam?
        \item What is the free spectral range for the \nth{1} order at \SI{600}{\nano\meter}?
    \end{enumerate}
\solution

\problem{}
    For a fiber optic probe with core and cladding refractive indices of \num{1.50} and \num{1.48}, respectively, and $\theta_i = \ang{28}$, calculate:
    \begin{enumerate}
        \item $\theta_r$
        \item NA
    \end{enumerate}
\solution
    \part 
        \begin{align*}
            \sin \theta_r &= \frac{n_1}{n_2} \sin \theta_i \\
            \theta_r &= \arcsin \left( \frac{n_1}{n_2} \sin \theta_i \right) \\
            &= \arcsin \left( \frac{1.50}{1.48} \sin \ang{28} \right) \\
            \theta_r &= \ang{28.4}
        \end{align*}
    \part 
        \begin{align*}
            \mathrm{N.A.} &= \sqrt{n_1^2 - n_2^2} \\
            &= \sqrt{\num{1.50}^2 - \num{1.48}^2} \\
            \mathrm{N.A.} &= \num{0.244}
        \end{align*}

\problem{}
    A monochromator has the following specifications:
    \begin{itemize}
        \item reciprocal linear dispersion = \SI{1.5}{\nano\meter\per\milli\meter}
        \item focal length = \SI{320}{\milli\meter}
        \item f/\# = \num{4.6}
        \item grating size: \num{68} $\times$ \SI{68}{\milli\meter}
        \item groove density = \SI{1800}{grooves \per\milli\meter}
    \end{itemize}
    Calculate the following at \SI{633}{\nano\meter} assuming the \nth{1} order is used:
    \begin{enumerate}
        \item Angular dispersion
        \item Linear dispersion
        \item Slit width to obtain a \SI{5}{\nano\meter} geometric spectral bandpass
    \end{enumerate}
\solution

\end{document}
