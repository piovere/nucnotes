\title{Homework \#3}
\author{J.R. Powers-Luhn}
\date{2018/04/06}

\documentclass{hw}
\usepackage{amsmath}
\usepackage[load=addn]{siunitx}

\sisetup{
    %exponent-to-prefix = true,
    zero-decimal-to-integer
}
    
\begin{document}

\problem{}

Draw the energy level diagram for a 4-level laser and explain in words how it works.

\solution

\setlength{\unitlength}{1cm}
\begin{picture}(10,6)
    %Energy Levels
    \put(1,5){$E_3$}
    \put(1.75,5){\line(1,0){3.5}}
    \put(9,4){$E_2$}
    \put(5.25,4){\line(1,0){3.5}}
    \put(9,2){$E_1$}
    \put(5.25,2){\line(1,0){3.5}}
    \put(1,1){$E_0$}
    \put(1.75,1){\line(1,0){7}}

    %Excitation E0->E3
    \put(2.25,1){\vector(0,1){4}}
    \put(1,3){{Pump}}

    %Fast decay E3->E2
    %Replace with squiggly line!
    \put(3.5,5){\vector(3,-1){3}}
    \put(6,4.4){{Fast decay, nonradiative}}

    %Laser action E2->E1
    \put(4,2.9){{Laser action}}
    \put(6.5,4){\vector(0,-1){2}}

    %Fast decay E1->E0
    %Replace with squiggly line!
    \put(5.5,2){\vector(0,-1){1}}
    \put(6,1.4){{Fast decay, nonradiative}}
\end{picture}

An energy source excites the lasing material from the ground state, $E_0$ to the excited state, $E_3$. The material then quickly relaxes to the $E_2$ state. Material in the $E_1$ state quickly decays to the $E_0$ state, where it can be excited $E_0 \xrightarrow{\text{Excitation}} E_3 \xrightarrow{\text{Fast decay}} E_2$. This forms the population inversion such that $N_{E_2} > N_{E_1}$. At this point, when a photon of energy $h\nu_{32}$ passes near the lasing material in state $E_2$, it stimulates an emission of a second photon of the same energy (``stimulated emission''). This leaves the atom in state $E_1$, where it non-radiatively relaxes to $E_0$ and can be excited again.

\problem{}

Calculate the (a) peak output and (b) average power for a laser that produces \SI{12}{\pico\second} pulses at a repetition rate of \SI{1000}{\kilo\hertz} with an energy of \SI{100}{\micro\joule\per pulse}.

\solution

\part
    \begin{align*}
        \SI{100}{\micro\joule} &= \frac{\SI{100e-6}{\joule\per pulse}}{\SI{12e-12}{\second\per pulse}} \\
        &= \SI{8}{\mega\watt}
    \end{align*}

\part
    \begin{align*}
        \SI{100e-6}{\joule\per pulse} \times \SI{1000e3}{\per\second} &= \SI{100}{\watt}
    \end{align*}

\problem{}

Briefly explain how a Q-switch works and the effect it has on the laser output.

\solution

A q-switched laser operates by first exciting the lasing medium (as with any other laser). Photons that are emitted during this process are kept out of the lasing medium, however, resulting in a low quality-factor ("q-factor"). This means that the excited atoms do not undergo stimulated emission, resulting in a larger population inversion than in continuous wave lasers. At some point, the mechanism separating the emitted photons from the lasing medium is disabled and the photons are allowed to enter the lasing medium--the q-factor is switched from low to high. This means that the larger population in the excited state can produce photons all at once, leading to a high-intensity pulse. Q-switch lasers therefore produce high-intensity pulses at comparatively low repetition rates.

\problem{}

Explain how CW, pulsed, and Q-switched laser pulses differ from each other.

\solution

\part{CW}

``Continuous Wave'' lasers output a near-constant intensity beam. The output beam is continuous with respect to time, and the energy output is in the \si{\milli\watt} to \si{\watt} range.

\part{Pulsed}

Pulsed lasers emit all of their energy in the amount of time required for relaxation of the laser medium. 

\part{Q-switched}

A q-switched laser 

\problem{}

Define the three $\chi$ terms which arise from the polarization in nonlinear optics and identify what frequency they oscillate at.

\solution

Nonlinear optical effects arise from especially intense electric fields. At low field intensities, the material is essentially unaffected by the beam electric field. At higher intensities/applied electric fields the material can be polarized and induce nonlinear effects:
$$
    P = \epsilon_0 \left( \chi^{(1)} E^{(1)} + \chi^{(2)} E^{(2)} + \chi^{(3)} E^{(3)} + ... \right)
$$

\part{$\chi^{\left(1\right)}$}

\begin{itemize}
    \item applies at low electric fields
    \item oscillates at incident frequency $\omega$
    \item light is refracted normally
\end{itemize}

\part{$\chi^{\left(2\right)}$}

\begin{itemize}
    \item only significant at high irradiances
    \item doubles frequency ($\omega \rightarrow 2\omega$)
    \item same direction as incident laser, still monochromatic
\end{itemize}

\part{$\chi^{\left(3\right)}$}

\begin{itemize}
    \item triples frequency ($\omega \rightarrow 3\omega$)
\end{itemize}

\problem{}

A Nd:YVO4 laser has two output wavelengths in the IR wavelength region, \SIlist{1064;1342}{\nano\meter}. We would like to use non-linear optics to create the harmonics of the \SI{1342}{\nano\meter} line. Identify the following wavelengths:
\begin{enumerate}
    \item Fundamental,
    \item Second harmonic, and
    \item Third harmonic.
\end{enumerate}

\solution



\problem{}

Describe how a CCD works.

\solution



\problem{}

What is the difference between random and non-random noise and is each one fundamental or non-fundamental?

\solution



\problem{}

Identify the following types of noise as either fundamental (F) or non-fundamental (NF):
\begin{enumerate}
    \item Shot noise,
    \item Pink noise,
    \item Interference,
    \item Dark current noise,
    \item Readout noise, and
    \item Impulse noise.
\end{enumerate}

\solution



\problem{}

Identify and define the 3 types of atomic spectroscopy.

\solution

\part
Absorption-- incident light of intensity $I_0(\lambda)$ is sent through the sample and the output light is measured at intensity of $I^\prime(\lambda)$. The ratio $\frac{I^\prime}{I_0}$ gives the absorbance of the sample at each wavelength examined.

\part
Emission-- the sample is excited by some process, e.g. heating into a plasma. This excites the electrons orbiting the atoms, which then relax, emitting light characteristic of the atom. These emissions are measured.

\part
Fluorescence-- the sample absorbs light of one wavelength and emits at another wavelength.

\end{document}