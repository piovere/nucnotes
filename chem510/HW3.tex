\documentclass{hw}
    \usepackage{amsmath}
    \usepackage{siunitx}
    \usepackage[super]{nth}
    \usepackage[version=4]{mhchem}
    
    \sisetup{
        exponent-to-prefix = true,
        zero-decimal-to-integer
    }
    
    \title{Homework \#3}
    \author{J.R. Powers-Luhn}
    \date{2018/04/06}
    
\begin{document}

\problem{}

Draw the energy level diagram for a 4-level laser and explain in words how it works.

\solution



\problem{}

Calculate the (a) peak output and (b) average power for a laser that produces \SI{12}{\pico\second} pulses at a repetition rate of \SI{1000}{\kilo\hertz} with an energy of \SI{100}{\micro\joule\per pulse}.

\solution

\part{Peak output}
    \begin{align*}
        \SI{100}{\micro\joule} &= \frac{\SI{100e-6}{\joule\per pulse}}{\SI{12e-12}{\second\per pulse}} \\
        &= \SI{8}{\mega\watt}
    \end{align*}

\part{Average power}
    \begin{align*}
        \SI{100e-6}{\joule\per pulse} \times \SI{1000e3}{\per\second} &= \SI{100}{\watt}
    \end{align*}

\problem{}

Briefly explain how a Q-switch works and the effect it has on the laser output.

\solution



\problem{}

Explain how CW, pulsed, and Q-switched laser pulses differ from each other.

\solution



\problem{}

Define the three $\chi$ terms which arise from the polarization in nonlinear optics and identify what frequency they oscillate at.

\solution



\problem{}

A Nd:YVO4 laser has two output wavelengths in the IR wavelength region, \SIlist{1064;1342}{\nano\meter}. We would like to use non-linear optics to create the harmonics of the \SI{1342}{\nano\meter} line. Identify the following wavelengths:
\begin{enumerate}
    \item Fundamental,
    \item Second harmonic, and
    \item Third harmonic.
\end{enumerate}

\solution



\problem{}

Describe how a CCD works.

\solution



\problem{}

What is the difference between random and non-random noise and is each one fundamental or non-fundamental?

\solution



\problem{}

Identify the following types of noise as either fundamental (F) or non-fundamental (NF):
\begin{enumerate}
    \item Shot noise,
    \item Pink noise,
    \item Interference,
    \item Dark current noise,
    \item Readout noise, and
    \item Impulse noise.
\end{enumerate}

\solution



\problem{}

Identify and define the 3 types of atomic spectroscopy.

\solution

\part{Awesome}
Lorem ipsum dolor simet.

\part{LIBS}
Lorem ipsum dolor simet.

\part{Dumb}
Lorem ipsum dolor simet.

\end{document}