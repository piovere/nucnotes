\documentclass[10pt,landscape]{article}
\usepackage{multicol}
\usepackage{calc}
\usepackage{ifthen}
\usepackage[landscape]{geometry}
\usepackage{amsmath,amsthm,amsfonts,amssymb}
\usepackage{color,graphicx,overpic}
\usepackage{hyperref}
\usepackage[load=addn]{siunitx}


\pdfinfo{
  /Title (chem580 Cheat Sheet 2)
  /Creator (TeX)
  /Producer (pdfTeX 1.40.0)
  /Author (J.R. Powers-Luhn)
  /Subject (chem580 Cheat Sheet 1)
  /Keywords (pdflatex, latex,pdftex,tex)}

% This sets page margins to .5 inch if using letter paper, and to 1cm
% if using A4 paper. (This probably isn't strictly necessary.)
% If using another size paper, use default 1cm margins.
\ifthenelse{\lengthtest { \paperwidth = 11in}}
    { \geometry{top=.5in,left=.5in,right=.5in,bottom=.5in} }
    {\ifthenelse{ \lengthtest{ \paperwidth = 297mm}}
        {\geometry{top=1cm,left=1cm,right=1cm,bottom=1cm} }
        {\geometry{top=1cm,left=1cm,right=1cm,bottom=1cm} }
    }

% Turn off header and footer
\pagestyle{empty}

% Redefine section commands to use less space
\makeatletter
\renewcommand{\section}{\@startsection{section}{1}{0mm}%
                                {-1ex plus -.5ex minus -.2ex}%
                                {0.5ex plus .2ex}%x
                                {\normalfont\large\bfseries}}
\renewcommand{\subsection}{\@startsection{subsection}{2}{0mm}%
                                {-1explus -.5ex minus -.2ex}%
                                {0.5ex plus .2ex}%
                                {\normalfont\normalsize\bfseries}}
\renewcommand{\subsubsection}{\@startsection{subsubsection}{3}{0mm}%
                                {-1ex plus -.5ex minus -.2ex}%
                                {1ex plus .2ex}%
                                {\normalfont\small\bfseries}}
\makeatother

% Define BibTeX command
\def\BibTeX{{\rm B\kern-.05em{\sc i\kern-.025em b}\kern-.08em
    T\kern-.1667em\lower.7ex\hbox{E}\kern-.125emX}}

% Don't print section numbers
\setcounter{secnumdepth}{0}


\setlength{\parindent}{0pt}
\setlength{\parskip}{0pt plus 0.5ex}

%My Environments
\newtheorem{example}[section]{Example}
% -----------------------------------------------------------------------

\begin{document}
\raggedright
\footnotesize
\begin{multicols}{2}


% multicol parameters
% These lengths are set only within the two main columns
%\setlength{\columnseprule}{0.25pt}
\setlength{\premulticols}{1pt}
\setlength{\postmulticols}{1pt}
\setlength{\multicolsep}{1pt}
\setlength{\columnsep}{2pt}

\begin{center}
     \Large{\underline{chem580 exam 1}} \\
\end{center}

\section{Binding Energy}
\begin{align*}
    B=15.56A-17.23A^{2/3}-0.72Z^2A^{-1/3}-23.285\left(A-2Z\right)^2A^{-1}+11A^{-1/2}
\end{align*}
\begin{align*}
    B=931 \left(1.00783Z + 1.00867N -M  \right)
\end{align*}

\section{Nuclear Radius}
\begin{align*}
    R(\si{cm})=\num{1.4e-13}A^{1/3}
\end{align*}

\section{Magic Numbers}
2, 8, 20, 28, 50, 82, 126 (also 118 for $p^+$)

\section{AMU}
\SI{1}{\amu}=\SI{931}{\mega\electronvolt}

\section{Coulomb Barrier}
\begin{align*}
    E_C=1.11 \frac{\left( A + A^\prime \right)}{A}
         \frac{Z Z^\prime}{\left(A^{1/3} + A^{\prime 1/3} \right)}
\end{align*}

\section{Specific Activity}
A is mass number, T is half-life in days
\subsection{\si[per=frac]{\milli\curie\per\milli\gram}}
$\frac{\num{1.3e8}}{AT}$
\subsection{\si[per=frac]{\mega\becquerel\per\milli\gram}}
$\frac{\num{4.8e6}}{AT}$

\section{Equilibrium}
\subsection{Secular}
\begin{align*}
  \lambda_d N_d = \lambda_p N_p \left( 1- \mathrm{e}^{-\lambda d t} \right)
\end{align*}

\section{Beta Recoil}
\begin{align*}
    E_{Max}=E_m \frac{m_e}{m_e+M_D}
\end{align*}

\section{Alpha Range}
\begin{align*}
    R_{air} = 0.31 E_\alpha^{1.5} \\
    R_{?} = R_{air} * \rho_{Air} / \rho_{?} \\
    E_\alpha = E_{decay} \frac{m_{Daughter}}{m_\alpha+m_{Daughter}} \\
\end{align*}
$\alpha$ produces \num{30000} ion pairs per cm in air

\section{Beta Ranges}
\begin{align*}
    E_{avg} \approx E_{max} / 3 \\
    E_{max}^{daughter} = E_\beta \frac{m_\beta}{m_\beta+M_{daughter}} \\
    R(mg/cm^2)=
    \begin{cases}
        543 E_m - 133 & E_m \geq \SI{0.8}{\mega\electronvolt} \\
        407 E_m^{1.38} & E_m < \SI{0.8}{\mega\electronvolt}
    \end{cases}
\end{align*}
Scattering increases with Z and surface density up to about 1/5 of the range

\section{Compton}
\begin{align*}
    E_{s} = \frac{E_{tot}}{ 1 + \frac{E_{tot}}{m_e}\left( 1 - \cos{\theta_\gamma} \right) }
\end{align*}

% You can even have references
\rule{0.3\linewidth}{0.25pt}
\scriptsize
\bibliographystyle{abstract}
\bibliography{refFile}
\end{multicols}
\end{document}