\documentclass{article}
\usepackage[version=4]{mhchem}
\usepackage{siunitx}
\usepackage{fancyhdr}
\pagestyle{fancy}

\author{J.R. Powers-Luhn}
\date{2016/01/25}
\title{Chemistry 580 Notes}

\rhead{J.R. Powers-Luhn}

\begin{document}

\section{Atomic Structure}

\subsection{Atoms}

\subsubsection{Elements and atoms}

\subsubsection{Atomic structure}

\subsubsection{Problems}

\subsubsection{Atomic (nuclidic) properties}

\subsubsection{Atom types}

\subsubsection{Problems}

\subsubsection{Occurrences}

\subsubsection{Even and odd generalizations}

\subsubsection{Magic numbers (nuclear shells)}

\subsubsection{Stable-atom systematics}

\subsubsection{Problems}

\newpage

\section{Radioactivity}

\subsection{Radioatoms or Radionuclides}

\subsubsection{Types of instability}
Typical $\alpha$ particle emission energies are $<\SI{10}{\mega\electronvolt}$, meaning that relativistic effects can be ignored. The decay energy is expressed by (Shultis, 95): $$ E_D = Q_\alpha \left[ \frac{M_\alpha}{M_D + M_\alpha} \right] \approx Q_\alpha \left[\frac{A_\alpha}{A_D + A_\alpha} \right] $$

\subsubsection{Neutron-rich radionuclides}
Because there are three particles emitted in $ \beta^- $ decay (daughter nucleus, electron, and electron-antineutrino), there is no unique solution to the energies of the resultant particles. The results are bound at the high end by the antineutrino having zero kinetic energy (Shultis, 97).

\newpage

\subsubsection{Neutron-deficient radionuclides}
Generally the process of electron capture leaves the nucleus in an excited state, implying that a gamma must be emmitted sometime later (Shultis, 89).

\end{document}
