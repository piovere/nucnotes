\documentclass{article}
\usepackage[version=4]{mhchem}
\usepackage{siunitx}

\author{J.R. Powers-Luhn}
\date{2016/01/25}
\title{Section 2.1 - 2.2}

\begin{document}

\section{Radioactivity}

\subsection{Radioatoms or Radionuclides}

\subsubsection{Types of instability}
Typical $\alpha$ particle emission energies are $<\SI{10}{\mega\electronvolt}$, meaning that relativistic effects can be ignored. The decay energy is expressed by (Shultis, 95): $$ E_D = Q_\alpha \left[ \frac{M_\alpha}{M_D + M_\alpha} \right] \approx Q_\alpha \left[\frac{A_\alpha}{A_D + A_\alpha} \right] $$

\subsubsection{Neutron-rich radionuclides}
Because there are three particles emitted in $ \beta^- $ decay (daughter nucleus, electron, and electron-antineutrino), there is no unique solution to the energies of the resultant particles. The results are bound at the high end by the antineutrino having zero kinetic energy (Shultis, 97).

\end{document}