\documentclass{hw}
\usepackage{amsmath}
\usepackage{nuc}
\usepackage{physics}
\usepackage{cancel}

\author{J.R. Powers-Luhn}
\date{2018/09/18}
\title{Homework \#3}

\begin{document}

\problem{}
    Use the product differentiation rule to show that the conservative and non-conservative forms of the 1D spherical equation are identical.

\solution
    Non-conservative form:
    \begin{align*}
    	\mu \pdv{\Psi (r, \mu, E)}{r} + \frac{1-\mu^2}{r} \pdv{\Psi (r, \mu, E)}{\mu} +\sigma_t(r, E) \Psi (r, \mu, E) &= q(r, \mu, E)
    \end{align*}

    Conservative form:
    \begin{align*}
    	\frac{\mu}{r^2} \pdv{\left[ r^2 \Psi(r, \mu, E) \right]}{r} + \pdv{}{\mu} \left[ \frac{\left( 1 - \mu^2 \right) \Psi(r, \mu, E)}{r} \right]
    		\sigma_t (r, E) \Psi ( r, \mu, E ) &= q(r, \mu, E) \\
    	\frac{\mu}{r^2} \left[ 2 r \Psi ( r, \mu, E )  r^2 \Psi ( r, \mu, E ) \right] + 
    		\frac{1}{r} \left[ -2 \mu \Psi ( r, \mu, E ) + \left( 1 - \mu^2 \right) \pdv{\Psi ( r, \mu, E )}{\mu} \right] + 
    		\sigma_t \Psi ( r, \mu, E ) &= q(r, \mu, E) \\
    	\cancel{\frac{2 \mu}{r} \Psi ( r, \mu, E )} + \mu \pdv{\Psi ( r, \mu, E )}{r} - \cancel{\frac{2 \mu}{r} \Psi ( r, \mu, E )} +
    		\frac{1 - \mu^2}{r} \pdv{\Psi ( r, \mu, E )}{\mu} + \sigma_t \Psi ( r, \mu, E ) &= q(r, \mu, E) \\
    	\mu \pdv{\Psi (r, \mu, E)}{r} + \frac{1-\mu^2}{r} \pdv{\Psi (r, \mu, E)}{\mu} +\sigma_t(r, E) \Psi (r, \mu, E) = q(r, \mu, E)
    \end{align*}

\problem{}
    Use the product differentiation rule to show that the conservative and non-conservative forms of the 1D cylindrical equation are identical.

\solution
    Non-conservative form:
    \begin{align*}
    	\mu \pdv{\Psi( r, \omega, \xi, E )}{r} - \frac{\eta}{r} \pdv{\Psi( r, \omega, \xi, E )}{\omega} + \sigma_t(r, E) \Psi( r, \omega, \xi, E ) &= q(r, \omega, \xi, E)
    \end{align*}

    Conservative form:
    \begin{align*}
    	\frac{\mu}{r} \pdv{\left[ r \Psi( r, \omega, \xi, E ) \right]}{r} - \frac{1}{r} \pdv{\left[ \eta \Psi( r, \omega, \xi, E ) \right]}{\omega} +
    		\sigma_t ( r, E ) \Psi( r, \omega, \xi, E ) &= q(r, \omega, \xi, E) \\
    	
    \end{align*}

\problem{}
    Show that the white boundary condition is given by: $$ \Psi(\vec{r}_s, \hat{\Omega}, E, t) = 4 J^+_n(\vec{r}_s, E, t) $$

\solution
    

\end{document}