\documentclass{hw}
\usepackage{amsmath}
\usepackage{cancel}
\usepackage{nuc}

\author{J.R. Powers-Luhn}
\date{2018/09/04}
\title{Homework \#1}

\begin{document}

\problem{}
    For each of the assumptions listed on slide 1-9, give a physical situation for which the assumption may not be a good one.
    \begin{itemize}
        \item Particles are points
        \item Particles travel in straight lines, unaccelerated until they interact
        \item Particles don't hit other particles
        \item Collisions are resolved instantaneously
        \item Material properties are the same no matter what direction a particle approaches
        \item Composition, configuration, and material properties are known and constant in time
        \item Only the expected (mean) values of reaction rates are needed
    \end{itemize}

\solution
    Lorem ipsum dolor simet

\problem{}
    Use integration by parts and l'Hopital's rule to show that:
    \begin{align*}
        \lambda = \frac{\int_0^\infty x \sigma_t I(x) dx}{\int_0^\infty \sigma_t I(x) dx} = \frac{\int_0^\infty x \mathrm{e}^{-\sigma_t x} dx }{\int_0^\infty \mathrm{e}^{-\sigma_t x} dx} = \ldots = \frac{\frac{1}{\sigma_t^2}}{\frac{1}{\sigma_t}} = \frac{1}{\sigma_t} 
    \end{align*}

\solution
    Lorem ipsum dolor simet

\end{document}