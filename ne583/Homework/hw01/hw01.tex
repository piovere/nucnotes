\documentclass{hw}
\usepackage{amsmath}
\usepackage{cancel}
\usepackage{nuc}

\author{J.R. Powers-Luhn}
\date{2018/09/04}
\title{Homework \#1}

\begin{document}
%\maketitle

\problem{}
    For each of the assumptions listed on slide 1-9, give a physical situation for which the assumption may not be a good one.
    \begin{itemize}
        \item Particles are points
        \item Particles travel in straight lines, unaccelerated until they interact
        \item Particles don't hit other particles
        \item Collisions are resolved instantaneously
        \item Material properties are the same no matter what direction a particle approaches
        \item Composition, configuration, and material properties are known and constant in time
        \item Only the expected (mean) values of reaction rates are needed
    \end{itemize}

\solution
    \begin{itemize}
        \item In particle accelerators where the beam energy is high, the wavelength can be comparable to the radius of the target particles. At this point it is necessary to treat the nucleons as having volume.
        \item In the presence of large masses (e.g. neutron stars), spacetime is curved. This distorts the path of particles.
        \item In the Large Hadron Collider, protons and anti-protons collide.
        \item There is a small but measurable delay between the absorption of a neutron and subsequent emission or fission event. There is a further delay before the decay of fission fragments produces decay neutrons.
        \item Some detector materials are made of long polymer strands that are (relatively) widely spaced, linked by small CH molecules. A particle incident on these detectors would interact differently if it entered parallel to the chain compared to perpendicular.
        \item In a PWR, the macroscopic cross section of the moderator material changes with temperature, altering the material properties.
        \item In a material that has been work hardened, interstitial nuclei with high cross sections would not be uniformly distributed throughout the material. The interstitial sites would interact with neutrons differently from the bulk lattice.
    \end{itemize}

\problem{}
    Use integration by parts and l'Hopital's rule to show that:
    \begin{align*}
        \lambda = \frac{\int_0^\infty x \sigma_t I(x) dx}{\int_0^\infty \sigma_t I(x) dx} = \frac{\int_0^\infty x \mathrm{e}^{-\sigma_t x} dx }{\int_0^\infty \mathrm{e}^{-\sigma_t x} dx} = \ldots = \frac{\frac{1}{\sigma_t^2}}{\frac{1}{\sigma_t}} = \frac{1}{\sigma_t} 
    \end{align*}

\solution
    \begin{align*}
        \lambda &= \frac{\int_0^\infty x \mathrm{e}^{-\sigma_t x} dx }{\int_0^\infty \mathrm{e}^{-\sigma_t x} dx} \\
        \lambda \int_0^\infty \mathrm{e}^{-\sigma_t x} dx &= \int_0^\infty x \mathrm{e}^{-\sigma_t x} dx \\
        u &= x \\
        \frac{du}{dx} &= 1 \\
        v &= -\frac{1}{\sigma_t} \mathrm{e}^{-\sigma_t x} \\
        \frac{dv}{dx} &= \mathrm{e}^{-\sigma_t x} \\
        \lambda \left(-\frac{1}{\sigma_t} \mathrm{e}^{-\sigma_t x} \Big|_0^\infty\right) &= -\frac{1}{\sigma_t} x \mathrm{e}^{-\sigma_t x}\Big|_0^\infty - \frac{1}{\sigma_t^2}\mathrm{e}^{-\sigma_t x}\Big|_0^\infty \\
        \lambda \frac{-1}{\sigma_t} \left( 0 - 1 \right) &= -\frac{1}{\sigma_t} x \mathrm{e}^{-\sigma_t x}\Big|_0^\infty - \frac{1}{\sigma_t^2} \left( 0 - 1 \right) \\
        \lambda \frac{1}{\sigma_t} &= \frac{1}{\sigma_t} \left( \frac{1}{\sigma_t} - \frac{x}{\mathrm{e}^{\sigma_t x}}\Big|_0^\infty \right) \\
        \lambda \frac{1}{\sigma_t} &= \frac{1}{\sigma_t} \left( \frac{1}{\sigma_t} - \left[ \lim_{x\to\infty}\frac{x}{\mathrm{e}^{\sigma_t x}} - \lim_{x\to0}\frac{x}{\mathrm{e}^{\sigma_t x}} \right] \right) \\
        \lambda \frac{1}{\sigma_t} &= \frac{1}{\sigma_t} \left( \frac{1}{\sigma_t} - \left[ \lim_{x\to\infty}\frac{x}{\mathrm{e}^{\sigma_t x}} - 0 \right] \right) \\
    \end{align*}

    To evaluate $\lim_{x\to\infty}\frac{x}{\mathrm{e}^{\sigma_t x}}$, we use l'Hopital's rule.

    \begin{align*}
        \lim_{x\to\infty}\frac{x}{\mathrm{e}^{\sigma_t x}} &= \lim_{x\to\infty}\frac{\frac{d}{dx}x}{\frac{d}{dx}\mathrm{e}^{\sigma_t x}} \\
        &= \lim_{x\to\infty}\frac{1}{\sigma_t \mathrm{e}^{\sigma_t x}} \\
        &= 0
    \end{align*}

    Plugging this back in, we get:

    \begin{align*}
        \lambda \frac{1}{\sigma_t} &= \frac{1}{\sigma_t} \left( \frac{1}{\sigma_t} - \left[ 0 - 0 \right] \right) \\
        \lambda \frac{1}{\sigma_t} &= \frac{1}{\sigma_t^2} \\
        \lambda &= {\frac{1}{\sigma_t^2}} / {\frac{1}{\sigma_t}} \\
        \lambda &= \frac{1}{\sigma_t}
    \end{align*}

\end{document}