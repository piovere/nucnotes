\documentclass{hw}
\usepackage{amsmath}
\usepackage{nuc}

\author{J.R. Powers-Luhn}
\date{2018/09/11}
\title{Homework \#2}

\begin{document}

\problem{}
    Repeat the Lagrangian derivation using time, $t$, as the parameter instead of distance, $s$, along the direction of travel.

\solution
    Lorem ipsum dolor simet

\problem{}
    How would the equation look for a charged particle with stopping power (i.e., energy loss per unit distance) of $S(E)$?

\solution
    Lorem ipsum dolor simet

\problem{}
    Fermi developed his “age theory” by assuming that neutron scattering was a continuous process (instead of happening instantaneously at each collision). Using $\xi$ (average lethargy gain per collision), show that $S(E)=E \xi \sigma_s$.

\solution
    Lorem ipsum dolor simet

\end{document}