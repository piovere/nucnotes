\documentclass{article}
\usepackage{physics}
\usepackage{siunitx}
\usepackage{graphicx}
\usepackage{courier}
\usepackage{listings}

\title{NE583 Test 2}
\author{J.R. Powers-Luhn}
\date{November 20, 2018}

\begin{document}

\maketitle

\section{Problem 1}

\section{Problem 2}

\subsection{Normalization factor}

\begin{align*}
\int_6^7 \psi(E) \dd{E} &= \int_6^7 \frac{1}{E} \dd{E} \\
&= \log{7} - \log{6} \\
f &= 0.154151
\end{align*}

\subsection{Alpha}

\begin{align*}
\alpha = \frac{(A-1)^2}{(A+1)^2} = 1
\end{align*}

\subsection{Scattering Cross section}

\begin{align*}
\sigma_s^{gg} &= \int_6^7 \dd{E^\prime} \int_6^7 \dd{E} \frac{\sigma}{(1 - \alpha)E} \frac{\psi(E)}{f} \\
&= \frac{\sigma}{f} \int_6^7 \dd{E^\prime} \int_6^7 \frac{\dd{E}}{E^2} \\
&= \frac{\sigma}{f} \int_6^7 \dd{E^\prime} \left( \frac{-1}{7} - \frac{-1}{6} \right) \\
&= \frac{\sigma}{f} \left( \frac{1}{6} - \frac{1}{7} \right) \left( 7 - 6 \right) \\
&= \frac{\SI{20}{\barn}}{0.154151} \left( \frac{1}{6} - \frac{1}{7} \right) \left( 7 - 6 \right) \\
&= \SI{3.089}{\barn}
\end{align*}

\section{Problem 3}

	%\lstinputlisting[language=python]{"Problem 3.py"}
	\input{"Problem 3.tex"}

\section{Problem 4}

\section{Problem 5}

\includegraphics[width=\textwidth]{no5result}

The orange line in the graph above was calculated using the code in the file \texttt{test2no5.java}. It 
is compared with the pseudo-analytic curve (blue line) which was generated by calculating the 
distance from every point in the source region to every point in the top row (in units of mean free 
paths) and assuming $\mathrm{e^{-r}}$ absorption and $\frac{1}{r^2}$ spatial falloff.

The quadrature solution shows a much more peaked distribution as a result of ray effects.

\end{document}