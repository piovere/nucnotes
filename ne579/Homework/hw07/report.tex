\documentclass{IEEEtran}

\usepackage[utf8]{inputenc}
\usepackage{graphicx}
\usepackage{amsmath}
\usepackage{siunitx}
\usepackage{listings}
\usepackage[citestyle=ieee,sorting=none,bibencoding=utf8,backend=biber]{biblatex}
\usepackage{caption}
\usepackage{subcaption}
\usepackage{booktabs}

\usepackage{algorithm}
\usepackage[noend]{algpseudocode}

\graphicspath{{images/}}
\bibliography{bibliography}
\makeatletter
\def\BState{\State\hskip-\ALG@thistlm}
\makeatother

\author{J.R. Powers-Luhn}
\title{Locally Weighted Regression}
\date{October 25th, 2018}

\begin{document}
\maketitle

\begin{abstract}



\end{abstract}

\section{Introduction}



\section{Methodology}



\subsection{Body Composition}
A dataset of body measurements (body fat percentage, age, weight, height, adiposity index, and ten 
circumference measurements) was obtained \cite{Penrose1985}. Ridge regression was used to generate models to 
predict body fat percentage from the other predictor variables using the two methods described above to select 
the hyperparameter $\alpha$.

\section{Results}

\begin{tabular}{ccc}
\toprule
 Bandwidth &  Order & RMSE (\% bodyfat) \\
\midrule
    0 &  0.15 &             11.13 \\
      & 0.25 &            11.07 \\
      & 0.50 &              8.97 \\
     &  0.75 &               7.99 \\
     &  1.00 &             7.64 \\
    &   1.50 &               7.50 \\
    &   2.00 &              7.48 \\
\midrule
   1 &    0.15 &          15.64 \\
    &   0.25 &               7.53 \\
 &      0.50 &                7.48 \\
    &   0.75 &                7.95 \\
     &  1.00 &               8.07 \\
    &   1.50 &                8.08 \\
   &    2.00 &                8.14 \\
 \midrule
 2 &      0.15 &                 9.05 \\
   &    0.25 &                 18.73 \\
    &   0.50 &               13.89 \\
     &  0.75 &                  12.68 \\
    &   1.00 &                  13.11 \\
    &   1.50 &                 8.93 \\
    &   2.00 &                  7.65 \\
\bottomrule
\end{tabular}


\subsection{Model Selection}
The model that was selected was that produced by the L-curve method. Its performance was not significantly 
different from the model produced using the LOO method and the larger $\alpha$ value should result in greater 
stability. Both models significantly improved on the least squares model, which had an RMSE of \num{19.35}.

The selected model was evaluated on the validation set data, producing an RMSE of \num{7.78}\% body fat.
\section{Conclusions}

A model was generated to predict body fat percentage from body composition predictors. The error of this model was \num{7.78}\% body fat. This was comparable to the error produced by a partial least squares model (\num{4.35}\%) and showed improvement over principal component regression (\num{4.72}\%) and a best-guess linear regression (\num{4.79}\%). Further, the linear model required the calculation of inverse terms while the ridge regression was both straightforward to calculate (unlike PLS, no iterative algorithms were necessary) and did not require non-linear terms. Two methods of selecting the hyperparameter $\alpha$ were examined. Leave one out cross validation was easy to understand but computationally expensive and might be prohibitive for large datasets. Variations on this method (e.g. ``Leave K out'') are worth exploring if they lead to stable results. The L-curve method of determining $\alpha$ clearly demonstrates the bias/variance trade, making it attractive if potentially subjective. 

\printbibliography

\onecolumn
\section{Appendix}
Python code used to perform calculations and generate graphics.
\lstset{frame=single}
\lstinputlisting[language=Python]{Homework07.py}

\end{document}