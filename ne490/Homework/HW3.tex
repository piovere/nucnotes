\documentclass{hw}
\usepackage[version=4]{mhchem}
\usepackage{amsmath}
\usepackage{cancel}
\usepackage[load=addn]{siunitx}
\usepackage{nuc}

\author{J.R. Powers-Luhn}
\date{2017/09/19}
\title{Homework \#3}

\begin{document}

% \maketitle

\problem{}
	Estimate how close an $\ce{H_3O+}$ ion and a hydrated electron must be to interact
\solution
	\begin{align*} 
        R_e &= \SI{2.1}{\angstrom} \\
        R_\ce{H_3O+} &= \SI{0.3}{\angstrom} \\
        R_{rxn} &= R_e + R_\ce{H_3O+} \\
        &= \SI{2.4}{\angstrom}
    \end{align*}

\problem{}
    A \SI{50}{\centi\meter^3} sample of water is given a dose of \SI{50}{\milli\gray} 
    from \SI{10}{\kilo\electronvolt} electrons. If the yield of \ce{H_2O_2} is 
    $G = 1.81$ per \SI{100}{\electronvolt}, how many molecules of \ce{H_2O_2} are 
    produced in the sample?
\solution
    \begin{align*}
        \frac{
            \SI{50}{\joule}
        }{
            \SI{1e3}{\kilo\gram}
        } \times 
        \SI{50}{\centi\meter^3}
         \times 
        \frac{
            \SI{1}{\gram} \ce{H_2O}
        }{
            \SI{1}{cm^3} \ce{H_2O}
        }
         \times 
        \frac{
            \SI{1}{\kilo\gram}
        }{
            \SI{1e3}{\gram}
        }
         \times 
        \frac{
            \SI{1}{\electronvolt}
        }{
            \SI{1.6e-19}{\joule}
        } \\
        =\num{2.82e14}
    \end{align*}

\problem{}
    During the chemical stage of radiolysis of water, what is the only chemical reaction to add reactivity to the system?
\solution
    \begin{align*}
        \ce{H_3O^+} + \ce{e^-_{aq}} \rightarrow \ce{H_2O} + \ce{\bullet H}
    \end{align*}

\problem{}
    During what phase of the cell cycle is the genetic material replicated?
\solution
    Interphase

\end{document}
