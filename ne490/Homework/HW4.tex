\documentclass{hw}
\usepackage[version=4]{mhchem}
\usepackage{amsmath}
\usepackage{cancel}
\usepackage[load=addn]{siunitx}
\usepackage{nuc}

\author{J.R. Powers-Luhn}
\date{2017/10/24}
\title{Homework \#4}

\begin{document}

% \maketitle

\problem{}
Procedure necessary to grow cell culture in a Petri dish.

\solution
	\begin{align*} 
        R_e &= \SI{2.1}{\angstrom} \\
        R_\ce{H_3O+} &= \SI{0.3}{\angstrom} \\
        R_{rxn} &= R_e + R_\ce{H_3O+} \\
        &= \SI{2.4}{\angstrom}
    \end{align*}

\problem{}
Definition of cell survival, i.e. how will you determine if a particular cell survives the radiation dose to which it is exposed?

\solution

\problem{}
Procedure for accounting for cell death due to experiment conditions, i.e. how would you control for cell death not attributed to radiation exposure but rather experiment conditions.

\solution

\problem{}
Suppose your experiment budget allows for the measurement of only five Petri dish cultures. Using the D0 doses for cell lines we have discussed in class as a guide, what five radiation doses would you select to produce your curve. Please give a short justification for this decision.

\solution

\problem{}
Procedure for determining how many cells to seed in each Petri dish.

\solution

\problem{}
Method for determining how many cells in each Petri dish survive radiation dose.

\solution

\problem{}
Hypothesis the shape of your measured survival curve if the cell line was a human sperm cell (i.e. shoulder or no shoulder). What values would you hypothesis for the value of D0 and n the extrapolation number the human sperm cell line?

\solution

\problem{}
Hypothesis the shapes of your measured survival curve if the cell line was a human melanoma (i.e. shoulder or no shoulder). What values would you hypothesis for the value of D0 and n the extrapolation number the melanoma cell line?

\solution

\problem{}
Procedure for accounting for cell cycle effects. As we discussed in class the cell cycle can greatly affect cell radiosensitivity. Describe a method for synchronizing the phase of cycle for cells in culture.

\solution

\end{document}
