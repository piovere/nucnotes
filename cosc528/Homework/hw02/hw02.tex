\documentclass{hw}
\usepackage{amsmath}
\usepackage{cancel}
\usepackage{nuc}
\usepackage{graphicx}
\usepackage{siunitx}

\graphicspath{ {./}}

\newcommand{\mean}{\bar}

\author{J.R. Powers-Luhn}
\date{2018/09/18}
\title{Homework \#2}

\begin{document}
%\maketitle

\problem{3.5}
    Propose a three-level cascade where when one level rejects, the next one is used as in equation 3.10. How can we fix the $\lambda$ on different levels? 

    \begin{equation}
        \lambda_{ik} =
        \begin{cases}
        0 & i = k \\
        \lambda & i = K + 1 \\
        1 & otherwise

    \end{cases}
    \end{equation}

\solution
    In this example, there is some loss imposed at each tier of rejection. In the event that there are three cascading classes, $C_1, C_2, C_3$, the following outcomes are possible:
    
    \begin{enumerate}
        \item The item is of class $C_1$. Loss is zero.
        \item The item is rejected from $C_1$ (incurring loss $\lambda_1$) and is of class $C_2$. Loss is $\lambda_1$.
        \item The item is rejected from $C_1$ and $C_2$, incurring a loss of $\lambda_1 + \lambda_2$
    \end{enumerate}

    If $\lambda_1 + \lambda_2 \geq 1$ then there will be a preference to miscategorize into $C_2$ over correctly categorizing in $C_3$. Similarly, if $\lambda_1 > 1$ then there will be a preference to miscategorize into $C_1$ over correctly categorizing into $C_2$ or $C_3$. The values of $\lambda$ must be set such that items that are correctly classified incur a loss less than if they are incorrectly classified.

\problem{3.9}
    Show that as we move an item from the antecedent to the consequent, confidence can never increase: confidence(ABC → D) ≥ confidence(AB → CD).

\solution
    \begin{align*}
        confidence(A,B,C \rightarrow D) &\geq confidence(A,B \rightarrow C,D) \\
        P(D | A, B, C) &\geq P(C, D | A, B) \\
        \frac{P(A, B, C, D)}{P(A, B, C)} &\geq \frac{P(A, B, C, D)}{P(A, B)} \\
        \frac{1}{P(A, B, C)} &\geq \frac{1}{P(A, B)} \\
        \frac{1}{P(C)*P(A,B|C)} &\geq \frac{1}{P(A, B)} \\
        \frac{P(C)}{P(C)*P(C | A, B)*P(A, B)} &\geq \frac{1}{P(A, B)} \\
        \frac{1}{P(C | A, B)} &\geq 1
    \end{align*}

    Since $P(C | A, B)$ is a probability it is bound in the range $[0,1]$. This is trivially true except in the case when $P(C | A, B)=0$.

\problem{3.10}
    Associated with each item sold in basket analysis, if we also have a number indicating how much the customer enjoyed the product, for example, on a scale of 0 to 10, how can you use this extra information to calculate which item to propose to a customer?

\solution
    \textbf{Option 1}: A complicated Bayesian inference could be drawn by assigning each product, $i$ a rating value from \numrange{0}{12} and calculating
    $$
        P(r_i | r_{j \neq i})
    $$
    to predict the likely rating for product $i$. Products with high predicted ratings would be recommended. Ratings of $11$ and $12$ would correspond to ``product not purchased'' and ``product purchased but not rated''.

    \textbf{Option 2}: Alternatively, we could treat the rating for each product as a pseudo-continuous distribution and measure cross-corellation between product ratings. This would allow us to recommend products with ratings strongly correlated to products our customer had rated highly in the past.

\end{document}